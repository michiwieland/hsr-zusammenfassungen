\newdualentry{cve}
{CVE}
{Common Vulnerabilities and Exposures}
{
	Eine einheitlichen Namenskonvention für Sicherheitslücken.
}

\newdualentry{nvd}
{NVD}
{National Vulnerability Database}
{
	Ist eine Sammlung von Sicherheitslücken der NIST.
}



% Glossar entries
\newglossaryentry{gls-asset}
{
	name={Asset},
	description={Dinge die es zu schützen gilt. Alles was für die Organisation von Wert ist. (Informationen, Wissen, Software, Mobilien, Dienstleistungen, Reputation, Image)
	}
}

\newglossaryentry{gls-threat}
{
	name={Threat},
	description={Eine Bedrohung ist der potenzielle Schaden. z.B Im Internet sind tausende Viren verbreitet. Es besteht also immer die Gefahr einer Infektion des Asset.
	}
}

\newglossaryentry{gls-appliedthread}
{
	name={Applied Threat (Threat + Vulnerability)},
	description={Eine Gefährdung welche konkret über eine Schwachstelle einwirkt. z.B Der Benutzer hat keinen Virenschutz installiert. Somit können Viren konkret über diese Schwachstelle einen Schaden}
}

\newglossaryentry{gls-vulnerability}
{
	name={Vulnerability},
	description={Eine Schwachstelle bezeichnet die Schwäche einer Schutzmassnahme die durch eine oder mehrere Bedrohungen ausgenutzt werden kann. (Infrastruktur, Prozesse, Personen)
	}
}

\newglossaryentry{gls-control}
{
	name={Control},
	description={Mit Controls werden die Assets vor potentiellen Bedrohungen geschützt. Dies können organisatorische, technische, personelle oder infrastrukturelle Sicherheitsmassnahmen sein.
	}
}

\newdualentry{sop}
{SOP}
{Same Origin Policy}
{
	Ist ein Sicherheitskonzept, das clientseitigen Skriptsprachen untersagt, auf Objekte zuzugreifen, die von einer Webseite stammen. (andere Origin)
}